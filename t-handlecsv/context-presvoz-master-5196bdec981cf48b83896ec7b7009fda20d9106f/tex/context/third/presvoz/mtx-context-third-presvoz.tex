%D \module
%D   [       file=mtx-context-third-presvoz,
%D        version=2022.02.02,
%D          title=PresVoz,
%D       subtitle=Presentations with Voice,
%D         author=Pablo Rodríguez,
%D           date=\currentdate,
%D      copyright={Michal Vlásak \&\ Pablo Rodríguez}]
%C
%C This module is part of the \CONTEXT\ macro||package and is
%C copyrighted by its authors.

\starttext
\startlmtxmode
\def\Presentation{\getdocumentfilename{1}}

\startluacode
if file.extname(document.files[1]) == "" then
  document.main_file = file.addsuffix(document.files[1],"pdf")
else
  document.main_file = document.files[1]
end

if file.pathpart(document.main_file) == "" then
  document.main_filename = file.nameonly(document.main_file)
else
  document.main_filename = file.pathpart(document.main_file):gsub(io.fileseparator, "/") .. "/" .. file.nameonly(document.main_file)
end

document.original_times = document.main_filename .. "-times.txt"

document.uncompressed_audio = document.main_filename .. "-audio.wav"

document.audio_mp3 = document.main_filename .. "-audio.mp3"

document.audio_opus = document.main_filename .. "-audio.opus" -- opus might work later

document.compressed_audio = document.audio_mp3

--twolame is not an option: gives wrong sizes

if io.exists(document.uncompressed_audio) and not(io.exists(document.audio_mp3)) and os.which('lame') then
  os.execute('lame -m m -b 32 "' .. document.uncompressed_audio .. '" "' .. document.audio_mp3 .. '"')
elseif io.exists(document.uncompressed_audio) and not(io.exists(document.compressed_audio)) then
  if os.which('opusenc') then os.execute('opusenc --downmix-mono --bitrate 16 "' .. document.uncompressed_audio .. '" "' .. document.audio_opus .. '"') end
end

presvoz_name = document.main_filename .. "-presvoz.pdf"

if not(io.exists(document.original_times)) or not(io.exists(document.compressed_audio)) then
  context.enablemode({"missing-files"})
end
\stopluacode

\startmode[missing-files]
\setupinteractionscreen[option=max]
\setupbodyfont[pagella]
\startTEXpage[offset=2em, align=center]
  \bf\color[darkred]{{\bi PresVoz} requires\\
  {\tt\cldcontext{document.original_times}} and {\tt\cldcontext{document.uncompressed_audio}}\\
  to do its magic with {\tt\cldcontext{document.main_file}}.}
\stopTEXpage
\stopmode

\startnotmode[missing-files]
\startluacode
document.timeline_filename = document.main_filename .. "-times.js"

document.slides_swf = document.main_filename .. ".swf"

document.presentation_script = document.main_filename .. "-presentation.sc"

document.presentation_swf = document.main_filename .. "-presentation.swf"

document.presentation_filename = file.nameonly(document.main_file) .. "-presentation.swf"

document.first_slide_swf = document.main_filename .. "-first.swf"

document.loader_script = document.main_filename .. "-loader.sc"

document.loader_swf = document.main_filename .. "-loader.swf"

function document.transfer_metadata(name)
  local main_doc = lpdf.epdf.load(name)
  lpdf.addtocatalog("Lang", lpdf.unicode(main_doc.Catalog.Lang), lpdf.string(main_doc.Catalog.Lang))
  if main_doc.Catalog.Lang == "en" then
    document.separate_subtitle = ": "
   else
    document.separate_subtitle = ". "
  end
  if main_doc.Info.Subject ~= nil and main_doc.Info.Subject ~= "" then
    context.setupinteraction{ title = main_doc.Info.Title .. document.separate_subtitle .. main_doc.Info.Subject }
  else
    context.setupinteraction{ title = main_doc.Info.Title }
  end
  if main_doc.Catalog.Lang == "es" then
    document.presvoz_banner = "Hecho con «PresVoz»  🤞🐞"
   elseif main_doc.Catalog.Lang == "pt" then
    document.presvoz_banner = "Feito con «PresVoz»  🤞🐞"
   elseif main_doc.Catalog.Lang == "it" then
    document.presvoz_banner = "Fatto con «PresVoz»  🤞🐞"
   elseif main_doc.Catalog.Lang == "fr" then
    document.presvoz_banner = "Fait avec « PresVoz »  🤞🐞"
   elseif main_doc.Catalog.Lang == "de" then
    document.presvoz_banner = "Erezeugt mit »PresVoz«  🤞🐞"
   elseif main_doc.Catalog.Lang == "nl" then
    document.presvoz_banner = "Gedaan med „PresVoz”  🤞🐞"
   elseif main_doc.Catalog.Lang == "nl" then
    document.presvoz_banner = "Gedaan med „PresVoz”  🤞🐞"
   else
    document.presvoz_banner = "Done with “PresVoz”  🤞🐞"
  end
  context.setupinteraction{ subtitle = document.presvoz_banner .. " – https://www.presvoz.tk" }
  if main_doc.Info.Author ~= nil and main_doc.Info.Author ~= "" then
    context.setupinteraction{ author = main_doc.Info.Author }
  end
  if main_doc.Info.ModDate ~=nil and main_doc.Info.ModDate ~= "" then
    original_Doc_date =  main_doc.Info.ModDate:sub(3,6) .. "-" ..
      main_doc.Info.ModDate:sub(7,8) .. "-" ..
      main_doc.Info.ModDate:sub(9,10) .. "T" ..
      main_doc.Info.ModDate:sub(11,12) .. ":" ..
      main_doc.Info.ModDate:sub(13,14) .. ":" ..
      main_doc.Info.ModDate:sub(15,22):gsub("'", ":")
    directives.enable('backend.date=' .. original_Doc_date .. "'")
  end
  lpdf.epdf.unload(name)
end

document.transfer_metadata(document.main_file)
\stopluacode

\startbuffer[flash-presentation]
    \startluacode
    local raw_times = io.open(document.original_times)
    local swf_times = io.open(document.presentation_script, "w")

    swf_times:write([[# Presentation script, generated by PresVoz
    #
    # Copyright (C) 2003 Leonard Lin
    # Copyright (C) 2007-2008 Sergio Costas
    # Copyright (C) 2009-2021 Pablo Rodríguez
    #
    # This program is free software; you can redistribute it and/or
    # modify it under the terms of the GNU General Public License
    # as published by the Free Software Foundation; either version 3
    # of the License, or (at your option) any later version.
    #
    # This program is distributed in the hope that it will be useful,
    # but WITHOUT ANY WARRANTY; without even the implied warranty of
    # MERCHANTABILITY or FITNESS FOR A PARTICULAR PURPOSE. See the GNU
    # General Public License for more details.
    #
    # You should have received a copy of the GNU General Public License
    # along with this program; if not, see <http://www.gnu.org/licenses>.
    #
    # Additional permission under GNU GPL version 3 section 7
    #
    # As a special exception, if you create a document which uses this
    # script, and compile this script or a modified version of it in the
    # document, this script does not by itself cause the resulting document
    # to be covered by the GNU General Public License. This exception does
    # not however invalidate any other reasons why the document might be
    # covered by the GNU General Public License.

    .flash filename="]] .. document.presentation_swf .. [[" version=8 fps=12

    .swf slides "]] .. document.slides_swf .. [["
    .sound audio "]] .. document.audio_mp3 .. [["

    .action:
        var times=new Array();
        var maxtimes;
        var position;
        var mode;
        var sound_position;
        var sound_length;
        var show_slide;

        var contextual = new ContextMenu();
        function ssurl(){
            getURL("https://www.presvoz.tk", "_blank");
        }

        switch (System.capabilities.language) {
            case "en" :
                ss_message = "Done with PresVoz  🤞🐞";
                break;
            case "es" :
                ss_message = "Hecho con PresVoz  🤞🐞";
                break;
            case "gl" :
                ss_message = "Feito con PresVoz  🤞🐞";
                break;
            case "pt" :
                ss_message = "Feito con PresVoz  🤞🐞";
                break;
            case "it" :
                ss_message = "Fatto con PresVoz  🤞🐞";
                break;
            case "fr" :
                ss_message = "Fait avec PresVoz  🤞🐞";
                break;
            case "de" :
                ss_message = "Erezeugt mit PresVoz  🤞🐞";
                break;
            case "nl" :
                ss_message = "Gedaan med PresVoz  🤞🐞";
                break;
            case "el" :
                ss_message = "Kατασκευασμένος με PresVoz  🤞🐞";
                break;
            default:
                ss_message = "Made with PresVoz  🤞🐞";
                break;
            }

        contextual.hideBuiltInItems();
        ss_url = new ContextMenuItem(ss_message, ssurl);
        contextual.customItems.push(ss_url);
        _root.menu = contextual;

        snd = new Sound(this);
        snd.attachSound("audio");
        sound_length=snd.duration;
        position = 0;

        show_slide = false;
        times[0] = 0;]] .. "\n")

    line_number = 0
    while true do
        line = raw_times:read()
        if line == nil then break end
        if line:match("%D") == nil then
          line_number = line_number + 1
          print ("times["..line_number.."] = " .. line .. ";")
                 swf_times:write("    times["..line_number.."] = " .. line .. ";")
                 swf_times:write("\n")
        end
    end

    swf_times:write("\n" .. [[
        maxtimes = times.length - 1;

        function pause_or_play() {
            if (mode==0) {
                snd.start(snd.position/1000.0,0);
                mode=1;
            } else if (mode==1) {
                snd.stop();
                mode=0;
            }
        }

        function previous_slide() {
            if (mode<2) {
                if (mode==0) { mode=1; }
                snd.stop();
                if (position>0) {
                    sound_position=times[position-1];
                } else {
                    sound_position=0;
                }
                snd.start(sound_position/1000.0,0);
            }
        }

        function next_slide() {
            if (mode<2) {
                if (mode==0) { mode=1; }
                snd.stop();
                sound_position = times[position+1];
                if (sound_position>sound_length) {
                    sound_position=sound_length-500;
                }
                snd.start(sound_position/1000.0,0);
            }
        }

        function first_slide() {
            snd.stop();
            snd.start(0,0);
            slides.gotoAndStop(1);
            position=0;
            mode=1;
            show_slide=true;
        }

        function last_slide() {
            snd.stop();
            sound_position = times[maxtimes-1];
            slides.gotoAndStop(maxtimes);
            show_slide=false;
            snd.start(sound_position/1000.0,0);
        }

        function toogle_fullscreen() {
            if(Stage.displayState == "normal"){
                Stage.displayState="fullScreen";
            } else {
                Stage.displayState="normal";
            }
        }

        Mouse.onMouseDown = function() {
            if ((getTimer() - lastClick) < 400) {
                pause_or_play();
            } else {
                lastClick = getTimer();
            }
        };


        key_listen=new Object();

        key_listen.onKeyDown = function() {
            switch (Key.getCode()) {
                case 32:
                    pause_or_play();
                break;
                case 37:
                    previous_slide();
                break;
                case 38:
                    first_slide();
                break;
                case 39:
                    next_slide();
                break;
                case 40:
                    last_slide();
                break;
                case 70:
                    toogle_fullscreen();
                break;
            }
        };

        function check_time() {
            if (mode==1) {
                if ( sound_position + 50 > sound_length) {
                    snd.stop();
                    mode=2;
                    if(Stage.displayState == "fullScreen"){
                          Stage.displayState="normal";
                    }
                } else {
                    if ((snd.position>=times[position])&&(snd.position<times[position+1])) {
                        if (show_slide) {
                            slides.gotoAndStop(position+1);
                            show_slide=false;
                        }
                    } else {
                        if (position>maxtimes) {
                            snd.stop();
                            mode=2;
                        } else {
                            if (snd.position<times[position]) {
                                position-=1;
                            } else {
                                position+=1;
                            }
                            show_slide=true;
                        }
                    }
                }
            }
        }

            setInterval(check_time, 10);

           mode=1;
           snd.start(0,0);
           Stage["displayState"] = "fullScreen";
           Mouse.addListener(Mouse);
           Key.addListener(key_listen);

    .end

    .frame 1
        .put slides
        .stop slides

    .end
    ]])

    raw_times:close()
    swf_times:close()

    os.execute('pdf2swf -f -T 8 -s filloverlap -s linksopennewwindow "' .. document.main_file .. '" -o "' .. document.slides_swf .. '"')
    os.execute('swfc "' .. document.presentation_script .. '"')
    if io.exists(document.presentation_swf) then os.remove(document.presentation_script) os.remove(document.slides_swf) end
    \stopluacode
\stopbuffer

\startbuffer[flash-loader]
    \startluacode
    local raw_times = io.open(document.original_times)
    local loader_times = io.open(document.loader_script, "w")

    fig = figures.getinfo(document.main_file)
    function dimension_bp(dimension)
        return math.floor(dimension*number.dimenfactors.bp)
    end

    loader_times:write([[# Preloader script, generated by PresVoz
    #
    # Copyright (C) 2009-2021 Pablo Rodríguez
    #
    # This program is free software; you can redistribute it and/or
    # modify it under the terms of the GNU General Public License
    # as published by the Free Software Foundation; either version 3
    # of the License, or (at your option) any later version.
    #
    # This program is distributed in the hope that it will be useful,
    # but WITHOUT ANY WARRANTY; without even the implied warranty of
    # MERCHANTABILITY or FITNESS FOR A PARTICULAR PURPOSE. See the GNU
    # General Public License for more details.
    #
    # You should have received a copy of the GNU General Public License
    # along with this program; if not, see <http://www.gnu.org/licenses>.
    #
    # Additional permission under GNU GPL version 3 section 7
    #
    # As a special exception, if you create a document which uses this
    # script, and compile this script or a modified version of this script
    # into the document, this script does not by itself cause the resulting
    # document to be covered by the GNU General Public License.
    # This exception does not however invalidate any other reasons why
    # the document might be covered by the GNU General Public License.

    .flash filename="]] .. document.loader_swf  .. [[" version=8 fps=12"
    .font roman "]] .. resolvers.findfile("iRaccoonShowBold.otf") ..  [[" glyphs=" %()KB0123456789/:,.CPDNuk[]resntiocanbld!"
    .edittext newloader font=roman color=#ffff00 size=30% width=350 height=30 border=none align="center" noselect variable=load_text
    .box placeholder width=]] .. dimension_bp(fig.used.width) .. [[ height=]] .. dimension_bp(fig.used.height) .. "\n" .. [[
    .box fsarea fill=#ffffff00 color=#ffffff00 width=]] .. dimension_bp(fig.used.width) .. [[ height=]] .. dimension_bp(fig.used.height) .. "\n" .. [[
    .outline fscreen_outline: M 62.78125 18.703125 L 43.367188 23.734375 L 38.390625 43.09375 L 42.109375 39.375 L 48.671875 45.9375 L 65.625 29.039062 L 59.0625 22.421875 L 62.78125 18.703125 z M 199.71875 18.703125 L 203.4375 22.421875 L 196.875 29.039062 L 213.773438 45.9375 L 220.390625 39.375 L 224.109375 43.09375 L 219.132812 23.734375 L 199.71875 18.703125 z M 83.125 45.9375 C 73.429688 45.9375 65.625 53.742188 65.625 63.4375 L 65.625 133.4375 C 65.625 143.132812 73.429688 150.9375 83.125 150.9375 L 179.375 150.9375 C 189.070312 150.9375 196.875 143.132812 196.875 133.4375 L 196.875 63.4375 C 196.875 53.742188 189.070312 45.9375 179.375 45.9375 L 83.125 45.9375 z M 48.726562 150.9375 L 42.109375 157.5 L 38.390625 153.78125 L 43.367188 173.140625 L 62.78125 178.171875 L 59.0625 174.453125 L 65.625 167.835938 L 48.726562 150.9375 z M 213.773438 150.9375 L 196.875 167.835938 L 203.4375 174.453125 L 199.71875 178.171875 L 219.132812 173.140625 L 224.109375 153.78125 L 220.390625 157.5 L 213.773438 150.9375 z .end
    .outline fulls_outline: M 17.5 0 L 245 0 C 254.695312 0 262.5 7.804688 262.5 17.5 L 262.5 179.375 C 262.5 189.070312 254.695312 196.875 245 196.875 L 17.5 196.875 C 7.804688 196.875 0 189.070312 0 179.375 L 0 17.5 C 0 7.804688 7.804688 0 17.5 0 z .end
    .filled fscreen outline=fscreen_outline fill=#969696ff color=#969696ff
    .filled fulls outline=fulls_outline fill=#64646496 color=#64646496

    .button fscr_button
        .show fscreen as=idle
        .show fscreen as=area
        .show fscreen as=hover
        .show fscreen as=pressed
    .end

    .button fs_area
        .show fsarea as=idle
        .show fsarea as=area
        .show fsarea as=hover
        .show fsarea as=pressed
        .on_release:
            fullscreen_button._visible = 0;
            fs_area._visible = 0;
            Stage["displayState"] = "fullScreen";
            _root.load_text = "[Click to start]";
            _root.dlprogress_mvclip._visible = true;
            the_presentation = "./]] .. document.presentation_filename .. [[";
            this.loadMovie(the_presentation, this.placeholder);
        .end
    .end

    .sprite fullscreen_button
        .put fscreen
        .put fulls
    .end

    .swf first_slide "]] .. document.first_slide_swf .. [["
    .frame 1
        .put first_slide
        .stop first_slide
        .put fullscreen_button
        .stop fullscreen_button
        .put fs_area
        .stop fs_area

        .action:
            slide_height=Stage.height;
            slide_width=Stage.width;

            var contextual = new ContextMenu();
            function ssurl(){ getURL("https://www.presvoz.tk", "_blank"); }

        switch (System.capabilities.language) {
            case "en" :
                ss_message = "Done with PresVoz  🤞🐞";
                break;
            case "es" :
                ss_message = "Hecho con PresVoz  🤞🐞";
                break;
            case "gl" :
                ss_message = "Feito con PresVoz  🤞🐞";
                break;
            case "pt" :
                ss_message = "Feito con PresVoz  🤞🐞";
                break;
            case "it" :
                ss_message = "Fatto con PresVoz  🤞🐞";
                break;
            case "fr" :
                ss_message = "Fait avec PresVoz  🤞🐞";
                break;
            case "de" :
                ss_message = "Erezeugt mit PresVoz  🤞🐞";
                break;
            case "nl" :
                ss_message = "Gedaan med PresVoz  🤞🐞";
                break;
            case "el" :
                ss_message = "Kατασκευασμένος με PresVoz  🤞🐞";
                break;
            default:
                ss_message = "Made with PresVoz  🤞🐞";
                break;
            }

            contextual.hideBuiltInItems();
            ss_url = new ContextMenuItem(ss_message, ssurl);
            contextual.customItems.push(ss_url);
            _root.menu = contextual;

            counter_percent=(slide_width/800)*100;
            newloader._xscale=counter_percent;
            newloader._yscale=counter_percent;
            newloader._x = (slide_width-(newloader._width))/2;
            newloader._y = (slide_height-newloader._height)/2;

            button_percent = (slide_width/600)*100;
            fullscreen_button._xscale = button_percent;
            fullscreen_button._yscale = button_percent;
            fullscreen_button._x = (slide_width-(fullscreen_button._width))/2;
            fullscreen_button._y = (slide_height-fullscreen_button._height)/2;

            _root.createEmptyMovieClip("dlprogress_mvclip", 0);
            dlprogress_mvclip.createEmptyMovieClip("bar", 1);
            dlprogress_mvclip.createEmptyMovieClip("stroke", 2);

            with (dlprogress_mvclip.stroke) {
               lineStyle(0, 0xFF0000, 30);
               moveTo(0, 0);
               lineTo(300, 0);
               lineTo(300, 40);
               lineTo(0, 40);
               lineTo(0, 0);
            }
            with (dlprogress_mvclip.bar) {
               beginFill(0xFF0000, 30);
               moveTo(0, 0);
               lineTo(300, 0);
               lineTo(300, 40);
               lineTo(0, 40);
               lineTo(0, 0);
               endFill();
               _xscale = 0;
            }
            dlprogress_mvclip._x = (slide_width-dlprogress_mvclip._width)/2;
            dlprogress_mvclip._y = (slide_height-dlprogress_mvclip._height)/2;

            dlprogress_mvclip._visible = false;
                    fs_listen=new Object();

            fs_listen.onKeyDown = function() {
                if (Key.isDown(13)) {
                    if(Stage.displayState == "normal"){
                        Stage.displayState="fullScreen";
                    } else {
                        Stage.displayState="normal";
                    }
                }
            };

            Key.addListener(fs_listen);


        .end

        .sprite images
            .put placeholder alpha=0%
            .action:
            the_presentation = "./]] .. document.presentation_filename .. [[";
            first_loading = new Object();
            first_loading.onMouseDown = function () {
                    load_swf = new MovieClipLoader();
                    load_swf.loadClip(the_presentation, placeholder);
                    listen_loading = new Object();
            listen_loading.onLoadStart = function(target) {
                _root.dlprogress_mvclip.bar._xscale = 0;
                _starttime = gettimer();
            };
            listen_loading.onLoadError = function(){
                _root.load_text = "Presentation cannot be loaded!";
            };
            listen_loading.onLoadProgress = function(target, bytesLoaded, bytesTotal) {
               downldtime = (gettimer() - _starttime) / 1000;
               _root.load_text = (String (Math.round ((bytesLoaded * 100) / bytesTotal)) + "% (" + Math.round((bytesLoaded / 1024) / downldtime) + " KB/s)");
               _root.dlprogress_mvclip.bar._xscale = Math.round(bytesLoaded/bytesTotal*100);
               keeping_awake = setInterval(function(){}, 1000);
            };
            listen_loading.onLoadComplete = function(target) {
               clearInterval(keeping_awake);
               _root.load_text = "";
               _root.dlprogress_mvclip._visible = 0;
               _root.dlprogress_mvclip.bar._visible = 0;
               _root.first_slide._visible = 0;
            };
                        load_swf.addListener(listen_loading);
                };
                Mouse.addListener(first_loading);
            .end
        .end

        .put newloader
        .stop newloader
    .end]])

    raw_times:close()
    loader_times:close()

    os.execute('pdf2swf -f -p 1 -T 8 -s filloverlap -s linksopennewwindow "' .. document.main_file .. '" -o "' .. document.first_slide_swf .. '"')
    os.execute('swfc "' .. document.loader_script .. '"')
    if io.exists(document.loader_swf) then os.remove(document.loader_script) os.remove(document.first_slide_swf) end
    \stopluacode
\stopbuffer

\def\TimeLine#1%
    {\cldcontext{ document.timeline_filename }}

\def\SoundFile#1%
    {\cldcontext{ document.compressed_audio }}

\startbuffer[generate-timeline]
    \startluacode
    local raw_times = io.open(document.original_times)
    local js_times = io.open(document.timeline_filename, "w")

    js_times:write("\\startJSpreamble times used now")
    js_times:write("\n")
    js_times:write("var times = [];")
    js_times:write("\n")
    js_times:write("var maxtimes = times[times.length - 1];")
    js_times:write("\n")
    js_times:write("times[0] = 0;")
    js_times:write("\n")
    line_number = 0
    while true do
        line = raw_times:read()
        if line == nil then break end
        if line:match("%D") == nil then
          line_number = line_number + 1
          print ("times["..line_number.."] = " .. line .. ";")
                 js_times:write("times["..line_number.."] = " .. line .. ";")
                 js_times:write("\n")
        end
    end

    js_times:write("\\stopJSpreamble")

    raw_times:close()
    js_times:close()
    \stopluacode
\stopbuffer

\startmode[*first]
\getbuffer[generate-timeline]
\doiftext
  {\ctxlua{if os.which('pdf2swf') and os.which('swfc') then context("a") end}}
  {\getbuffer[flash-presentation]
   \getbuffer[flash-loader]}
\stopmode

\getfiguredimensions[\Presentation]

\setupexternalfigures[interaction=all]

\setupinteraction
    [state=start,
     color=,
     contrastcolor=,
     style=,
     focus=standard]

\useJSscripts[presvoz.mkxl]
\startnotmode[*first]
  \useJSscripts[\TimeLine{\Presentation}]
\stopnotmode

\definefontfamily
    [mainface]
    [rm]
    [Latin Modern Sans]

\definefontfamily
    [mainface]
    [ss]
    [Flechitas]

\setupbodyfont
    [mainface, 25pt]

\setupmakeup
    [page]
    [pagestate=start,
     style=\bfd,
     color=white,
     footerstate=start,
     align=center]

\definecolor[transparent-gray]
    [r=.45, g=.45, b=.45]

\defineviewerlayer[fullscreen][state=start]
\defineviewerlayer[unfullscreen][state=stop]
\defineviewerlayer[play][state=start]
\defineviewerlayer[pause][state=stop]

\definecolor[transparent][r=1, a=1, t=0]

\startbuffer[fullscreen-button]
  \startoverlay
    {\viewerlayer[fullscreen]{C}}
    {\viewerlayer[unfullscreen]{I}}
    {\goto{\color[transparent]{C}}[JS(SwitchFS{})]}
  \stopoverlay
\stopbuffer

\startbuffer[play-pause-button]
  \startoverlay
    {\viewerlayer[play]{R}}
    {\viewerlayer[pause]{H}}
    {\goto{\color[transparent]{C}}[JS(PlayPauseSound{})]}
  \stopoverlay
\stopbuffer

\def\SlideNavigationButtons{%
    \goto{F}[JS(GoToFirstSlide{})]\,%
    \goto{B}[JS(GoToPreviousSlide{})]\,%
    \inlinebuffer[play-pause-button]\,%
    \goto{S}[JS(StopSound{})]\,%
    \goto{N}[JS(GoToNextSlide{})]\,%
    \goto{L}[JS(GoToLastSlide{})]\quad
    \null\inlinebuffer[fullscreen-button]%
    }

\definelayer[player-buttons]
    [x=.5\dimexpr\figurewidth\relax,
     y=.9525\dimexpr\figureheight\relax,
     location=middle,
     state=repeat]

\definelayer[placesound]

\startbuffer[placesound]
  \definerenderingwindow[soundplace]
    [width=0pt, height=0pt, frame=on]

\doiftextelse{\ctxlua{if document.compressed_audio == document.audio_opus then context("opus") end}}
  {\userendering[mainsound][audio/opus][\SoundFile{\Presentation}][embed=yes, list=no]}
  {\userendering[mainsound][audio/mp3][\SoundFile{\Presentation}][embed=yes, list=no]}

  \placerenderingwindow[soundplace][mainsound]
\stopbuffer

\setupbackgrounds
    [page]
    [background={foreground, placesound, player-buttons}]


\setlayer[player-buttons]
    {\switchtobodyfont[sans, \cldcontext{tex.dimen.paperwidth/1000000}pt]\color[transparent-gray]{\SlideNavigationButtons}}

\resetpagenumber

\dorecurse{\noffigurepages}{\startTEXpage[pagestate=start]
\ifnum\recurselevel=1 \setlayer[placesound]{\getbuffer[placesound]}\fi
\externalfigure[\Presentation][page=\recurselevel]
\stopTEXpage}
\stopnotmode

\startluacode
luatex.wrapup(
  function()
    os.remove(presvoz_name)
    os.rename(tex.jobname .. ".pdf", presvoz_name)
  end
)
\stopluacode
\stoptext
\stoplmtxmode

\startmkivmode
\setupinteractionscreen[option=max]
\startTEXpage[offset=2em, align=center]
  \ssbf\color[darkgreen]{\LMTX\ required,\\I’m afraid.}
\stopTEXpage
\stopmkivmode
\stoptext
