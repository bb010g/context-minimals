% Copyright 2018-2023 by Romano Giannetti
% Copyright 2015-2023 by Stefan Lindner
% Copyright 2013-2023 by Stefan Erhardt
% Copyright 2007-2023 by Massimo Redaelli
%
% This file may be distributed and/or modified
%
% 1. under the LaTeX Project Public License and/or
% 2. under the GNU Public License.
%
% See the files gpl-3.0_license.txt and lppl-1-3c_license.txt for more details.

% This first line *shouldn't* be needed; it's here to work around a problem
% in ConTeXt TikZ support when it switch to LMTX. Kept here to make CI happy.
% See also https://github.com/circuitikz/circuitikz/issues/706
% See https://github.com/pgf-tikz/pgf/issues/1248#issuecomment-1486235591
% I have no clue on how to fix this; moving the following line *inside*
% t-circuitikz.tex does not work
\usemodule[pgfrcs,pgfmat]
\usemodule[circuitikz]

\starttext

A simple example to test the installation.

% removed all the "pseudo-SI commands". They were removed by ConTeXt,
% when it moved to LMTX (March 2023).
% For units you should use http://www.pragma-ade.nl/general/manuals/units-mkiv.pdf
\startcircuitikz[scale=1.5]
	\draw
  (0,2) to[I=1~mA] (2,2)
        to[R, l_=2~kΩ, *-*] (0,0)
        to[R, l_=2~kΩ] (2,0)
        to[V, v_=2~V] (2,2)
        to[cspst, l=$t_0$] (4,2) -- (4,1.5)
        to [generic, i=$i_1$, v=$v_1$] (4,-.5) -- (4,-1.5)
  (0,2) -- (0,-1.5) to[V, v_=4~V] (2,-1.5)
        to [R, l=1~kΩ] (4,-1.5)
  (5,2) node[dipchip, anchor=pin 1]{}
  (5,-2) node[flipflop JK, anchor=pin 1]{};

\stopcircuitikz

\stoptext
